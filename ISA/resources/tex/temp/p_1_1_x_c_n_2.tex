% Preamble

\documentclass{amsart}

%\usepackage{amssymb, amscd, amsfonts}
%\usepackage[mathscr]{eucal}
%\usepackage{graphicx}
\usepackage{lscape}
\usepackage{tikz}
\usepackage{scalefnt}
%\usepackage[applemac]{inputenc}
\usepackage[latin1]{inputenc}
%\usepackage[utf8x]{inputenc}
%\usepackage{verbatim}
%\usepackage[export]{adjustbox}
%\usepackage{lipsum}
\usepackage{float}% If comment this, figure moves to Page 2
\usepackage[pgf,outputdir={docgraphs/}]{dot2texi}
%\usepackage[active,tightpage]{preview}
\usetikzlibrary{shapes,automata,arrows}



%% Definitions of proclamation environments, commands, shorthands, et cetera

% Proclamation environments

\newtheorem{theorem}{Theorem}[section]
\newtheorem{lemma}[theorem]{Lemma}
\newtheorem{corollary}[theorem]{Corollary}
\newtheorem{proposition}[theorem]{Proposition}
\newtheorem{fact}[theorem]{Fact}

\theoremstyle{definition}
\newtheorem{observation}[theorem]{Observation}
\newtheorem{definition}[theorem]{Definition}
\theoremstyle{remark}
\newtheorem*{algoprob}{Problem}        % To typeset decision problems

\theoremstyle{remark}
\newtheorem*{algoinst}{Instance}       % To typeset decision problems

\theoremstyle{remark}
\newtheorem*{algoquest}{Question}       % To typeset decision problems

\theoremstyle{remark}
\newtheorem*{conjecture}{Conjecture}

\theoremstyle{definition}
\newtheorem{problem}{Problem}

\theoremstyle{remark}
\newtheorem*{question}{Question}

\theoremstyle{definition}
\newtheorem{remark}{Remark}

\theoremstyle{definition}
\newtheorem*{remarks}{Remarks}

\theoremstyle{definition}
\newtheorem{example}{Example}

\theoremstyle{remark}
\newtheorem{examples}{Examples}

\theoremstyle{definition}
\newtheorem{notation}[theorem]{Notation}

\theoremstyle{remark}
\newtheorem*{addnotation}{Additional Notation}

\theoremstyle{remark}
\newtheorem*{runningnotation}{Running Notation}

% Math Commands

\newcommand{\barecoeff}[2]{\gen\frac {} {} {0pt} {} {#1}{#2}}
\newcommand{\wstirl}[2]{\bigl\{\bigl\{\genfrac{}{}{0pt}{}{#1}{#2}\bigr\}\bigr\}}
\newcommand{\sstirl}[2]{ \genfrac {\{} {\}} {0pt} {} {#1}{#2} }
\newcommand{\wbinom}[2]{\bigl(\bigl(\genfrac{}{}{0pt}{}{#1}{#2}\bigr)\bigr)}


% Math Operators

\DeclareMathOperator{\fan}{\rm Fan}
\DeclareMathOperator{\chrom}{\chi} \DeclareMathOperator{\sign}{\rm
signum} \DeclareMathOperator{\partitio}{\rm p}
\DeclareMathOperator{\bbpartitio}{\rm p}
\DeclareMathOperator{\bbcappartitio}{\mathscr{P}}
\DeclareMathOperator{\Sub}{\rm Sub}
\DeclareMathOperator{\Parts}{\mathscr{P}}
\DeclareMathOperator{\MVParts}{\mathscr{MVP}}
\DeclareMathOperator{\Hom}{\textsc Hom}
\DeclareMathOperator{\Var}{\textxsc Var}
\DeclareMathOperator{\F}{\textxsc Form}
\DeclareMathOperator{\Eval}{\textsc Eval}
\DeclareMathOperator{\rank}{\rm rank}
\DeclareMathOperator{\pos}{\rm pos} \DeclareMathOperator{\lin}{\rm
lin} \DeclareMathOperator{\relint}{\rm relint}
\DeclareMathOperator{\con}{\rm conv}
\DeclareMathOperator{\aff}{\rm aff}
\DeclareMathOperator{\supp}{\rm supp}
\DeclareMathOperator{\eff}{\rm eff} \DeclareMathOperator{\st}{\rm
star} \DeclareMathOperator{\cstar}{\rm cstar}
\DeclareMathOperator{\lnk}{\rm link}
\DeclareMathOperator{\rel}{\rm relint}
\DeclareMathOperator{\Dom}{\rm Dom} \DeclareMathOperator{\Cod}{\rm
Cod} \DeclareMathOperator{\Ker}{\rm Ker}
\DeclareMathOperator{\Ran}{\rm Range}
\DeclareMathOperator{\GRan}{\rm GRange}
\DeclareMathOperator{\Dim}{\rm Dim}
\DeclareMathOperator{\vertices}{\rm vert}
\DeclareMathOperator{\Rad}{\mathscr{R}}
\DeclareMathOperator{\Spec}{\bf Spec}
\DeclareMathOperator{\Max}{\bf MaxSpec}
\DeclareMathOperator{\Zero}{\mathbf{Z}}
\DeclareMathOperator{\V}{\mathbb{V}}
\DeclareMathOperator{\I}{\mathbb{I}}
\DeclareMathOperator{\E}{\mathbb{E}}
\DeclareMathOperator{\A}{\mathbb{A}}
\DeclareMathOperator{\Aut}{\rm Aut}
\DeclareMathOperator{\FaceId}{\rm FaceId}
\DeclareMathOperator{\FaceR}{\rm FaceR}
\DeclareMathOperator{\GroupId}{\rm GroupId}
\DeclareMathOperator{\GroupR}{\rm GroupR}
\DeclareMathOperator{\den}{\rm den} \DeclareMathOperator{\lcm}{\rm
lcm}

\DeclareMathOperator{\tree}{\rm Tree}

% Commands

%% Temporary Annotations

\newcommand{\commento}[1]{\marginpar{\footnotesize \flushleft{ #1}}}

\newcommand{\remove}[1]{}

%% Sets of Numbers
\newcommand{\N}{{\mathbb{N}}}
\newcommand{\Prim}{{\mathbb{P}}}
\newcommand{\Z}{{\mathbb{Z}}}
\newcommand{\Q}{{\mathbb{Q}}}
\newcommand{\R}{{\mathbb{R}}}
\newcommand{\C}{{\mathbb{C}}}
\newcommand{\CC}{{\mathscr{C}}}
\newcommand{\lex}{{\overrightarrow{\oplus}}}
\newcommand{\MVN}{{\mathbb{M}\mathbb{V}\mathbb{N}}}
\newcommand{\MVZ}{{\mathbb{M}\mathbb{V}\mathbb{Z}}}
\newcommand{\MVQ}{{\mathbb{M}\mathbb{V}\mathbb{Q}}}
\newcommand{\MVR}{{\mathbb{M}\mathbb{V}\mathbb{R}}}
\newcommand{\MVC}{{\mathbb{M}\mathbb{V}\mathbb{C}}}

%% Frequent Bores
\newcommand{\asc}{\mathrm{asc}}
\newcommand{\des}{\mathrm{des}}

\newcommand{\QQ}{ \mathbb{Q} } % -- Rational numbers
\newcommand{\NN}{\mathbb{N}} % --- Natural numbers
\newcommand{\DD}{\mathcal{D}}
\newcommand{\LL}{\mathcal{L}}

\renewcommand{\SS}{\mathcal{S}}
\newcommand{\vbold}{\mathbf{v}}
\newcommand{\wbold}{\mathbf{w}}
\newcommand{\mss}{{\sf m}}
\newcommand{\M}{{\mathscr{M}}}
\newcommand{\G}{{\mathbf{\Gamma}}}
\newcommand{\Hats}{{\mathbf{H}}}
\newcommand{\lgp}{{$\ell$-group }}
\newcommand{\lgpO}{{$\ell$-group. }}
\newcommand{\lgps}{{$\ell$-groups }}
\newcommand{\lgpsO}{{$\ell$-groups. }}
\newcommand{\Fl}{\mathscr{A}}
\newcommand{\Fln}{\mathscr{A}_n}
\newcommand{\MV}{MV algebra }
\newcommand{\MVO}{MV algebra. }
\newcommand{\MVs}{MV algebras }
\newcommand{\MVsO}{MV algebras. }
\newcommand{\FMV}{\mathscr{M}}
\newcommand{\FMVn}{\mathscr{M}_n}
\newcommand{\Pro}{\textsc{BasProj}}
\newcommand{\In}{\textsc{Input: $\, \,$}}
\newcommand{\Out}{\textsc{Output: $\, \,$}}
\newcommand{\Y}{\textsc{Yes }}
\newcommand{\No}{\textsc{No }}
\newcommand{\pvar}{\vec{\pi}}
\newcommand{\li}{\langle \langle}
\newcommand{\ri}{\rangle \rangle}
\newcommand{\MM}{\mathcal{M}}
\renewcommand{\div}{{\, \, \rm div \,\,}}


%% Funny Names
\newcommand{\Lu}{{\L}u\-ka\-s\-i\-e\-w\-icz }
\newcommand{\Lus}{{\L}u\-ka\-s\-i\-e\-w\-icz's }
\newcommand{\Bez}{Bezout }

% Body
\begin{document}

\title{ISA Software v.1.3}

%%% Authors

%\author{O. M. D'Antona}

%%% Date

\date{\today}

%%% Addresses

%\address[O. M. D'Antona]{Dipartimento di Informatica e Comunicazione,
%via Comelico 39/41, I-20135 Milano, Italy}

%\email{dantona@dico.unimi.it}

\keywords{sample.tex}

%\subjclass[2000]{Primary: . Secondary: .}
%%% Abstract

%\begin{abstract}
%Evoluzione? 
%\end{abstract}

\maketitle 

%%% stampa test famiglie grafi

\section{ Caso di studio : Grafo $P_1^{( 1)}\times C_5^{( 2)}$  } 



\bigskip \begin{definition}
 Un grafo (non orientato e finito) � una coppia ordinata $(V,E)$ dove $V$ � un insieme finito ed $E$ � un multiinsieme di coppie non ordinate di elementi di $V$. L'insieme V contiene i vertici del grafo ed $E$ i suoi lati. Per un generico grafo $G$, l'insieme dei suoi vertici � indicato con $V(G)$ e quello dei suoi lati con $E(G)$.
 \end{definition} \par


\bigskip La struttura dati con la quale si � scelto di memorizzare il grafo � la matrice di adicenza.\par


\bigskip \begin{definition} 
La matrice di adiacenza di un grafo $G$ i cui vertici siano $v_1,v_2, \dots ,v_n$ � una matrice $A(G)=[a(i,j)]$  simmetrica di ordine $n\times n$ in cui si pone: 
\end{definition} 
$$a(i,j)=\left\{\begin{tabular}{ll} 
1& se $(v_i,v_j)\in E(G)$ \\ 
0& altrimenti 
\end{tabular}\right. 
 $$ 

\par


\bigskip Di seguito viene mostrata invece la lista di adiacenza che permette una pi� facile lettura delle adiacenze:\par


$$\left\{
\begin{tabular}{ll} 
	$(1;1) \longrightarrow\  (1;2), (1;3), (1;4), (1;5),  $\\
	$(1;2) \longrightarrow\  (1;1), (1;3), (1;4), (1;5),  $\\
	$(1;3) \longrightarrow\  (1;1), (1;2), (1;4), (1;5),  $\\
	$(1;4) \longrightarrow\  (1;1), (1;2), (1;3), (1;5),  $\\
	$(1;5) \longrightarrow\  (1;1), (1;2), (1;3), (1;4),  $\\
\end{tabular}
\right.$$
$$\scalefont{0.3}\begin{tikzpicture}
		 [,>=stealth',shorten >=1pt,auto,node distance=1.4cm,thick,main node/.style={circle,draw,font=\sffamily\small}] 
	\node[main node] (1) {1.1}; 
	\node[main node] (2) [right of=1] {1.2}; 
	\node[main node] (3) [right of=2] {1.3}; 
	\node[main node] (4) [right of=3] {1.4}; 
	\node[main node] (5) [right of=4] {1.5}; 
 \path[every node/.style={font=\sffamily\small}] 	(2) edge  node[above] {} (1) 
	(3) edge  node[above] {} (2) 
	(4) edge  node[above] {} (3) 
	(5) edge  node[above] {} (4) 
	(1) edge [bend right] node[above] {} (3) 
	(2) edge [bend right] node[above] {} (4) 
	(3) edge [bend right] node[above] {} (5) 
	(4) edge [bend right] node[above] {} (1) 
	(5) edge [bend right] node[above] {} (2) 
(1) edge [bend left] node[above] {} (5); \end{tikzpicture}$$ 

\bigskip In forma circolare diventa:\par

\begin{figure}[H]

	\centering
		\begin{dot2tex}[circo,tikz] 
			strict graph { 
				graph[center=true, margin=0.2, nodesep=0.1, ranksep=0.3] 
				node[shape=circle, fontname="Courier-Bold", fontsize=10, width=0.4, height=0.4, fixedsize=true] 
				edge[dir=none, arrowsize=0.6, arrowhead=vee] 

				 "1;1" -- "1;2"  ; 
				 "1;1" -- "1;3"  ; 
				 "1;1" -- "1;4"  ; 
				 "1;1" -- "1;5"  ; 
				 "1;2" -- "1;1"  ; 
				 "1;2" -- "1;3"  ; 
				 "1;2" -- "1;4"  ; 
				 "1;2" -- "1;5"  ; 
				 "1;3" -- "1;1"  ; 
				 "1;3" -- "1;2"  ; 
				 "1;3" -- "1;4"  ; 
				 "1;3" -- "1;5"  ; 
				 "1;4" -- "1;1"  ; 
				 "1;4" -- "1;2"  ; 
				 "1;4" -- "1;3"  ; 
				 "1;4" -- "1;5"  ; 
				 "1;5" -- "1;1"  ; 
				 "1;5" -- "1;2"  ; 
				 "1;5" -- "1;3"  ; 
				 "1;5" -- "1;4"  ; 
			} 
		\end{dot2tex} 
\end{figure} 

 

\bigskip Con le famiglie di grafi $C$ vogliamo indicare dei circuiti \emph{veri e propri} in cui, oltre all'arco che collega il primo nodo con l'ultimo, abbiamo anche archi delle potenze dei cammini orizzontali che possono collegarsi ai nodi precedenti rispetto ai nodi dai quali partono. \par
\subsection{Calcolo insiemi indipendenti con metodo forza bruta} \



\bigskip \begin{definition}Un insieme indipendente di un grafo � un insieme di vertici non adiacenti del grafo.\end{definition}\par


\bigskip Definiamo $T(n,k)$ il numero di $k$-sottoinsiemi indipendenti di Grafo $P_1^{( 1)}\times C_5^{( 2)}$ . \\Ecco alcuni valori\par
$$\begin{tabular}{c | r r r r r}
$T(n,k)$&$k=0$&1&2&3&4\\ \hline 
$0$&1\\ 
$1$&1&1\\ 
$2$&1&2\\ 
$3$&1&3\\ 
$4$&1&4\\ 
$5$&1&5\\ 
$6$&1&6&3\\ 
$7$&1&7&7\\ 
$8$&1&8&12\\ 
$9$&1&9&18&3\\ 
$10$&1&10&25&10\\ 
$11$&1&11&33&22\\ 
$12$&1&12&42&40&3\\ 
\end{tabular}$$



\bigskip Seguono le successioni delle antidiagonali, della somma delle righe e dei valori massimali di $k$ per cui esistono insiemi indipendenti:\par

$$\begin{tabular}{c | r r r r r r r r r r r r r}
$n$&$0$&1&2&3&4&5&6&7&8&9&10&11&12\\ \hline
$AD_n$& 1& 1& 2& 3& 4& 5& 6& 7& 11& 16& 22& 29& 40\\ \hline
$RS_n$& 1& 2& 3& 4& 5& 6& 10& 15& 21& 31& 46& 67& 98\\ \hline
$K_n$& 0& 1& 1& 1& 1& 1& 2& 2& 2& 3& 3& 3& 4\\ \hline

\end{tabular}$$


\bigskip \emph{Ricerca delle bijezioni disabilitata per questa stampa.}\par


\bigskip \textbf{Wilf}: Non possiamo usare il metodo di Wilf per trovare la Fgo delle somme delle righe in quanto il grafo \`e un circuito.\par
\subsection{Automa} \

L'automa che riconosce (tutte e sole) le stringhe che corrispondono agli insiemi indipendenti di $C_n^{(2)}$ diventa: 

$$\begin{tikzpicture}[->,>=stealth',shorten >=1pt,auto,node distance=2.cm, 
thick,main node/.style={circle,fill=blue!20,draw,font=\sffamily},main node2/.style={circle,fill=green!20,draw,font=\sffamily},main node3/.style={circle,fill=yellow!20,draw,font=\sffamily}] 
	 \node[initial,initial text=,state,accepting] (1) {S}; 
	 \node[initial text=,state,accepting] (2) [right of=1] {$G_1$}; 
	 \node[initial text=,state,accepting] (3) [right of=2] {$G_2$}; 
	 \node[initial text=,state,accepting] (4) [right of=3] {$G_3$}; 
	 \node[initial text=,state] (5) [right of=4] {$G_4$}; 
	 \node[initial text=,state] (6) [right of=5] {$G_5$}; 
	 \node[initial text=,state,accepting] (7) [below of=2] {$E_1$}; 
	 \node[initial text=,state,accepting] (8) [right of=7] {$E_2$}; 
	 \node[initial text=,state,accepting] (9) [right of=8] {$E_3$}; 
	 \node[initial text=,state] (10) [right of=9] {$E_4$}; 
	 \node[initial text=,state,accepting] (11) [right of=10] {$E_5$}; 
	 \node[initial text=,state,accepting] (12) [below of=8] {$F_1$}; 
	 \node[initial text=,state,accepting] (13) [right of=12] {$F_2$};  
	 \node[initial text=,state,accepting] (14) [right of=13] {$F_3$}; 
	 \node[initial text=,state,accepting] (15) [right of=14] {$F_4$}; 
	 \path[every node/.style={font=\sffamily\small}] 
	 (1) edge[] node[] {$a$} (2) 
	 (2) edge[] node[] {$e$} (3) 
	 (3) edge[] node[] {$e$} (4) 
	 (4) edge[loop above] node[left=1pt] {e} (4) 
	 (4) edge[] node[] {$a$} (5) 
	 (5) edge[] node[] {$e$} (6) 
	 (6) edge[bend right] node[above] {$e$} (4) 
	 (7) edge[] node[] {$e$} (8) 
	 (8) edge[] node[] {$e$} (9) 
	 (9) edge[loop above] node[left=1pt] {e} (9) 
	 (9) edge[] node[] {$a$} (10) 
	 (10) edge[] node[] {$e$} (11) 
	 (11) edge[bend right] node[above] {$a$} (9) 
	 (12) edge[] node[] {$e$} (13) 
	 (13) edge[loop above] node[left=1pt] {e} (13) 
	 (13) edge[] node[] {$a$} (14) 
	 (14) edge[] node[] {$e$} (15) 
	 (15) edge[bend right] node[above] {$e$} (13) 
	 (1) edge[bend right] node[above] {$e$} (12) 
	 (12) edge[bend right] node[above] {$a$} (7) 
	 ; 
\end{tikzpicture}$$ 


\bigskip Il sistema lineare diventa:\par
\bigskip \bigskip  \noindent\(\pmb{\text{eqs}=\{}\\ 
\pmb{s\text{==}t*\text{f1}+t*\text{g1}+1,}\\ 
\pmb{\text{f1}==t*\text{f2}+t*\text{e1}+1,}\\ 
\pmb{\text{f2}==t*\text{f2}+t*\text{f3}+1,}\\ 
\pmb{\text{f3}==t*\text{f4}+1,}\\ 
\pmb{\text{f4}==t*\text{f2}+1,}\\ 
\pmb{\text{e1}==t*\text{e2}+1,}\\ 
\pmb{\text{e2}==t*\text{e3}+1,}\\ 
\pmb{\text{e3}==t*\text{e3}+t*\text{e4}+1,}\\ 
\pmb{\text{e4}==t*\text{e5},}\\ 
\pmb{\text{e5}==t*\text{e3}+1,}\\ 
\pmb{\text{g1}==t*\text{g2}+1,}\\ 
\pmb{\text{g2}==t*\text{g3}+1,}\\ 
\pmb{\text{g3}==t*\text{g3}+t*\text{g4}+1,}\\ 
\pmb{\text{g4}==t*\text{g5},}\\ 
\pmb{\text{g5}==t*\text{g3}}\\ 
\pmb{\}}\) 

\bigskip \bigskip $$F(t)=\frac{-1-t-t^2+t^4+2 t^5}{-1+t+t^3}$$ 

\bigskip \bigskip $$1+2 t+3 t^2+4 t^3+5 t^4+6 t^5+10 t^6+15 t^7+21 t^8+31 t^9+O[t]^{10}$$ 


\bigskip 
\bigskip 
\bigskip 
\bigskip 
\bigskip 
















































































 
\end{document}