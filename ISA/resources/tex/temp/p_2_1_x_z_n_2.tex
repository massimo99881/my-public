% Preamble

\documentclass{amsart}

%\usepackage{amssymb, amscd, amsfonts}
%\usepackage[mathscr]{eucal}
%\usepackage{graphicx}
\usepackage{lscape}
\usepackage{tikz}
\usepackage{scalefnt}
%\usepackage[applemac]{inputenc}
\usepackage[latin1]{inputenc}
%\usepackage[utf8x]{inputenc}
%\usepackage{verbatim}
%\usepackage[export]{adjustbox}
%\usepackage{lipsum}
\usepackage{float}% If comment this, figure moves to Page 2
\usepackage[pgf,outputdir={docgraphs/}]{dot2texi}
%\usepackage[active,tightpage]{preview}
\usetikzlibrary{shapes,automata,arrows}



%% Definitions of proclamation environments, commands, shorthands, et cetera

% Proclamation environments

\newtheorem{theorem}{Theorem}[section]
\newtheorem{lemma}[theorem]{Lemma}
\newtheorem{corollary}[theorem]{Corollary}
\newtheorem{proposition}[theorem]{Proposition}
\newtheorem{fact}[theorem]{Fact}

\theoremstyle{definition}
\newtheorem{observation}[theorem]{Observation}
\newtheorem{definition}[theorem]{Definition}
\theoremstyle{remark}
\newtheorem*{algoprob}{Problem}        % To typeset decision problems

\theoremstyle{remark}
\newtheorem*{algoinst}{Instance}       % To typeset decision problems

\theoremstyle{remark}
\newtheorem*{algoquest}{Question}       % To typeset decision problems

\theoremstyle{remark}
\newtheorem*{conjecture}{Conjecture}

\theoremstyle{definition}
\newtheorem{problem}{Problem}

\theoremstyle{remark}
\newtheorem*{question}{Question}

\theoremstyle{definition}
\newtheorem{remark}{Remark}

\theoremstyle{definition}
\newtheorem*{remarks}{Remarks}

\theoremstyle{definition}
\newtheorem{example}{Example}

\theoremstyle{remark}
\newtheorem{examples}{Examples}

\theoremstyle{definition}
\newtheorem{notation}[theorem]{Notation}

\theoremstyle{remark}
\newtheorem*{addnotation}{Additional Notation}

\theoremstyle{remark}
\newtheorem*{runningnotation}{Running Notation}

% Math Commands

\newcommand{\barecoeff}[2]{\gen\frac {} {} {0pt} {} {#1}{#2}}
\newcommand{\wstirl}[2]{\bigl\{\bigl\{\genfrac{}{}{0pt}{}{#1}{#2}\bigr\}\bigr\}}
\newcommand{\sstirl}[2]{ \genfrac {\{} {\}} {0pt} {} {#1}{#2} }
\newcommand{\wbinom}[2]{\bigl(\bigl(\genfrac{}{}{0pt}{}{#1}{#2}\bigr)\bigr)}


% Math Operators

\DeclareMathOperator{\fan}{\rm Fan}
\DeclareMathOperator{\chrom}{\chi} \DeclareMathOperator{\sign}{\rm
signum} \DeclareMathOperator{\partitio}{\rm p}
\DeclareMathOperator{\bbpartitio}{\rm p}
\DeclareMathOperator{\bbcappartitio}{\mathscr{P}}
\DeclareMathOperator{\Sub}{\rm Sub}
\DeclareMathOperator{\Parts}{\mathscr{P}}
\DeclareMathOperator{\MVParts}{\mathscr{MVP}}
\DeclareMathOperator{\Hom}{\textsc Hom}
\DeclareMathOperator{\Var}{\textxsc Var}
\DeclareMathOperator{\F}{\textxsc Form}
\DeclareMathOperator{\Eval}{\textsc Eval}
\DeclareMathOperator{\rank}{\rm rank}
\DeclareMathOperator{\pos}{\rm pos} \DeclareMathOperator{\lin}{\rm
lin} \DeclareMathOperator{\relint}{\rm relint}
\DeclareMathOperator{\con}{\rm conv}
\DeclareMathOperator{\aff}{\rm aff}
\DeclareMathOperator{\supp}{\rm supp}
\DeclareMathOperator{\eff}{\rm eff} \DeclareMathOperator{\st}{\rm
star} \DeclareMathOperator{\cstar}{\rm cstar}
\DeclareMathOperator{\lnk}{\rm link}
\DeclareMathOperator{\rel}{\rm relint}
\DeclareMathOperator{\Dom}{\rm Dom} \DeclareMathOperator{\Cod}{\rm
Cod} \DeclareMathOperator{\Ker}{\rm Ker}
\DeclareMathOperator{\Ran}{\rm Range}
\DeclareMathOperator{\GRan}{\rm GRange}
\DeclareMathOperator{\Dim}{\rm Dim}
\DeclareMathOperator{\vertices}{\rm vert}
\DeclareMathOperator{\Rad}{\mathscr{R}}
\DeclareMathOperator{\Spec}{\bf Spec}
\DeclareMathOperator{\Max}{\bf MaxSpec}
\DeclareMathOperator{\Zero}{\mathbf{Z}}
\DeclareMathOperator{\V}{\mathbb{V}}
\DeclareMathOperator{\I}{\mathbb{I}}
\DeclareMathOperator{\E}{\mathbb{E}}
\DeclareMathOperator{\A}{\mathbb{A}}
\DeclareMathOperator{\Aut}{\rm Aut}
\DeclareMathOperator{\FaceId}{\rm FaceId}
\DeclareMathOperator{\FaceR}{\rm FaceR}
\DeclareMathOperator{\GroupId}{\rm GroupId}
\DeclareMathOperator{\GroupR}{\rm GroupR}
\DeclareMathOperator{\den}{\rm den} \DeclareMathOperator{\lcm}{\rm
lcm}

\DeclareMathOperator{\tree}{\rm Tree}

% Commands

%% Temporary Annotations

\newcommand{\commento}[1]{\marginpar{\footnotesize \flushleft{ #1}}}

\newcommand{\remove}[1]{}

%% Sets of Numbers
\newcommand{\N}{{\mathbb{N}}}
\newcommand{\Prim}{{\mathbb{P}}}
\newcommand{\Z}{{\mathbb{Z}}}
\newcommand{\Q}{{\mathbb{Q}}}
\newcommand{\R}{{\mathbb{R}}}
\newcommand{\C}{{\mathbb{C}}}
\newcommand{\CC}{{\mathscr{C}}}
\newcommand{\lex}{{\overrightarrow{\oplus}}}
\newcommand{\MVN}{{\mathbb{M}\mathbb{V}\mathbb{N}}}
\newcommand{\MVZ}{{\mathbb{M}\mathbb{V}\mathbb{Z}}}
\newcommand{\MVQ}{{\mathbb{M}\mathbb{V}\mathbb{Q}}}
\newcommand{\MVR}{{\mathbb{M}\mathbb{V}\mathbb{R}}}
\newcommand{\MVC}{{\mathbb{M}\mathbb{V}\mathbb{C}}}

%% Frequent Bores
\newcommand{\asc}{\mathrm{asc}}
\newcommand{\des}{\mathrm{des}}

\newcommand{\QQ}{ \mathbb{Q} } % -- Rational numbers
\newcommand{\NN}{\mathbb{N}} % --- Natural numbers
\newcommand{\DD}{\mathcal{D}}
\newcommand{\LL}{\mathcal{L}}

\renewcommand{\SS}{\mathcal{S}}
\newcommand{\vbold}{\mathbf{v}}
\newcommand{\wbold}{\mathbf{w}}
\newcommand{\mss}{{\sf m}}
\newcommand{\M}{{\mathscr{M}}}
\newcommand{\G}{{\mathbf{\Gamma}}}
\newcommand{\Hats}{{\mathbf{H}}}
\newcommand{\lgp}{{$\ell$-group }}
\newcommand{\lgpO}{{$\ell$-group. }}
\newcommand{\lgps}{{$\ell$-groups }}
\newcommand{\lgpsO}{{$\ell$-groups. }}
\newcommand{\Fl}{\mathscr{A}}
\newcommand{\Fln}{\mathscr{A}_n}
\newcommand{\MV}{MV algebra }
\newcommand{\MVO}{MV algebra. }
\newcommand{\MVs}{MV algebras }
\newcommand{\MVsO}{MV algebras. }
\newcommand{\FMV}{\mathscr{M}}
\newcommand{\FMVn}{\mathscr{M}_n}
\newcommand{\Pro}{\textsc{BasProj}}
\newcommand{\In}{\textsc{Input: $\, \,$}}
\newcommand{\Out}{\textsc{Output: $\, \,$}}
\newcommand{\Y}{\textsc{Yes }}
\newcommand{\No}{\textsc{No }}
\newcommand{\pvar}{\vec{\pi}}
\newcommand{\li}{\langle \langle}
\newcommand{\ri}{\rangle \rangle}
\newcommand{\MM}{\mathcal{M}}
\renewcommand{\div}{{\, \, \rm div \,\,}}


%% Funny Names
\newcommand{\Lu}{{\L}u\-ka\-s\-i\-e\-w\-icz }
\newcommand{\Lus}{{\L}u\-ka\-s\-i\-e\-w\-icz's }
\newcommand{\Bez}{Bezout }

% Body
\begin{document}

\title{ISA Software v.1.3}

%%% Authors

%\author{O. M. D'Antona}

%%% Date

\date{\today}

%%% Addresses

%\address[O. M. D'Antona]{Dipartimento di Informatica e Comunicazione,
%via Comelico 39/41, I-20135 Milano, Italy}

%\email{dantona@dico.unimi.it}

\keywords{sample.tex}

%\subjclass[2000]{Primary: . Secondary: .}
%%% Abstract

%\begin{abstract}
%Evoluzione? 
%\end{abstract}

\maketitle 

%%% stampa test famiglie grafi

\section{ Caso di studio : Grafo $P_2^{( 1)}\times Z_8^{( 2)}$  } 



\bigskip \begin{definition}
 Un grafo (non orientato e finito) � una coppia ordinata $(V,E)$ dove $V$ � un insieme finito ed $E$ � un multiinsieme di coppie non ordinate di elementi di $V$. L'insieme V contiene i vertici del grafo ed $E$ i suoi lati. Per un generico grafo $G$, l'insieme dei suoi vertici � indicato con $V(G)$ e quello dei suoi lati con $E(G)$.
 \end{definition} \par


\bigskip La struttura dati con la quale si � scelto di memorizzare il grafo � la matrice di adicenza.\par


\bigskip \begin{definition} 
La matrice di adiacenza di un grafo $G$ i cui vertici siano $v_1,v_2, \dots ,v_n$ � una matrice $A(G)=[a(i,j)]$  simmetrica di ordine $n\times n$ in cui si pone: 
\end{definition} 
$$a(i,j)=\left\{\begin{tabular}{ll} 
1& se $(v_i,v_j)\in E(G)$ \\ 
0& altrimenti 
\end{tabular}\right. 
 $$ 

\par


\bigskip Di seguito viene mostrata invece la lista di adiacenza che permette una pi� facile lettura delle adiacenze:\par


$$\left\{
\begin{tabular}{ll} 
	$(1;1) \longrightarrow\  (2;1), (1;2), (2;2), (1;3),  $\\
	$(2;1) \longrightarrow\  (1;1), (1;2), (2;2), (2;3),  $\\
	$(1;2) \longrightarrow\  (1;1), (2;1), (2;2), (1;3), (2;3), (1;4),  $\\
	$(2;2) \longrightarrow\  (1;1), (2;1), (1;2), (1;3), (2;3), (2;4),  $\\
	$(1;3) \longrightarrow\  (1;1), (1;2), (2;2), (2;3), (1;4), (2;4), (1;5),  $\\
	$(2;3) \longrightarrow\  (2;1), (1;2), (2;2), (1;3), (1;4), (2;4), (2;5),  $\\
	$(1;4) \longrightarrow\  (1;2), (1;3), (2;3), (2;4), (1;5), (2;5), (1;6),  $\\
	$(2;4) \longrightarrow\  (2;2), (1;3), (2;3), (1;4), (1;5), (2;5), (2;6),  $\\
	$(1;5) \longrightarrow\  (1;3), (1;4), (2;4), (2;5), (1;6), (2;6), (1;7),  $\\
	$(2;5) \longrightarrow\  (2;3), (1;4), (2;4), (1;5), (1;6), (2;6), (2;7),  $\\
	$(1;6) \longrightarrow\  (1;4), (1;5), (2;5), (2;6), (1;7), (2;7), (1;8),  $\\
	$(2;6) \longrightarrow\  (2;4), (1;5), (2;5), (1;6), (1;7), (2;7), (2;8),  $\\
	$(1;7) \longrightarrow\  (1;5), (1;6), (2;6), (2;7), (1;8), (2;8),  $\\
	$(2;7) \longrightarrow\  (2;5), (1;6), (2;6), (1;7), (1;8), (2;8),  $\\
	$(1;8) \longrightarrow\  (1;6), (1;7), (2;7), (2;8),  $\\
	$(2;8) \longrightarrow\  (2;6), (1;7), (2;7), (1;8),  $\\
\end{tabular}
\right.$$
$$\scalefont{0.3}\begin{tikzpicture}
		 [,>=stealth',shorten >=1pt,auto,node distance=1.4cm,thick,main node/.style={circle,draw,font=\sffamily\small}] 
	\node[main node] (1) {1.1}; 
	\node[main node] (2) [below of=1] {2.1}; 
	\node[main node] (3) [right of=1] {1.2}; 
	\node[main node] (4) [right of=2] {2.2}; 
	\node[main node] (5) [right of=3] {1.3}; 
	\node[main node] (6) [right of=4] {2.3}; 
	\node[main node] (7) [right of=5] {1.4}; 
	\node[main node] (8) [right of=6] {2.4}; 
	\node[main node] (9) [right of=7] {1.5}; 
	\node[main node] (10) [right of=8] {2.5}; 
	\node[main node] (11) [right of=9] {1.6}; 
	\node[main node] (12) [right of=10] {2.6}; 
	\node[main node] (13) [right of=11] {1.7}; 
	\node[main node] (14) [right of=12] {2.7}; 
	\node[main node] (15) [right of=13] {1.8}; 
	\node[main node] (16) [right of=14] {2.8}; 
 \path[every node/.style={font=\sffamily\small}] 	(2) edge  node[above] {} (1) 
	(4) edge  node[above] {} (3) 
	(6) edge  node[above] {} (5) 
	(8) edge  node[above] {} (7) 
	(10) edge  node[above] {} (9) 
	(12) edge  node[above] {} (11) 
	(14) edge  node[above] {} (13) 
	(16) edge  node[above] {} (15) 
	(3) edge  node[above] {} (1) 
	(5) edge  node[above] {} (3) 
	(7) edge  node[above] {} (5) 
	(9) edge  node[above] {} (7) 
	(11) edge  node[above] {} (9) 
	(13) edge  node[above] {} (11) 
	(15) edge  node[above] {} (13) 
	(4) edge  node[above] {} (2) 
	(6) edge  node[above] {} (4) 
	(8) edge  node[above] {} (6) 
	(10) edge  node[above] {} (8) 
	(12) edge  node[above] {} (10) 
	(14) edge  node[above] {} (12) 
	(16) edge  node[above] {} (14) 
	(5) edge [bend right] node[above] {} (1) 
	(7) edge [bend right] node[above] {} (3) 
	(9) edge [bend right] node[above] {} (5) 
	(11) edge [bend right] node[above] {} (7) 
	(13) edge [bend right] node[above] {} (9) 
	(15) edge [bend right] node[above] {} (11) 
	(6) edge [bend right] node[above] {} (2) 
	(8) edge [bend right] node[above] {} (4) 
	(10) edge [bend right] node[above] {} (6) 
	(12) edge [bend right] node[above] {} (8) 
	(14) edge [bend right] node[above] {} (10) 
	(16) edge [bend right] node[above] {} (12) 
(3) edge [] node[above] {} (2)(4) edge [] node[above] {} (1)(5) edge [] node[above] {} (4)(6) edge [] node[above] {} (3)(7) edge [] node[above] {} (6)(8) edge [] node[above] {} (5)(9) edge [] node[above] {} (8)(10) edge [] node[above] {} (7)(11) edge [] node[above] {} (10)(12) edge [] node[above] {} (9)(13) edge [] node[above] {} (12)(14) edge [] node[above] {} (11)(15) edge [] node[above] {} (14)(16) edge [] node[above] {} (13); \end{tikzpicture}$$ \subsection{Calcolo insiemi indipendenti con metodo forza bruta} \



\bigskip \begin{definition}Un insieme indipendente di un grafo � un insieme di vertici non adiacenti del grafo.\end{definition}\par


\bigskip Definiamo $T(n,k)$ il numero di $k$-sottoinsiemi indipendenti di Grafo $P_2^{( 1)}\times Z_8^{( 2)}$ . \\Ecco alcuni valori\par
$$\begin{tabular}{c | r r r r r}
$T(n,k)$&$k=0$&1&2&3&4\\ \hline 
$0$&1\\ 
$1$&1&2\\ 
$2$&1&4\\ 
$3$&1&6&2\\ 
$4$&1&8&8\\ 
$5$&1&10&18&2\\ 
$6$&1&12&32&12\\ 
$7$&1&14&50&38&2\\ 
$8$&1&16&72&88&16\\ 
\end{tabular}$$



\bigskip Seguono le successioni delle antidiagonali, della somma delle righe e dei valori massimali di $k$ per cui esistono insiemi indipendenti:\par

$$\begin{tabular}{c | r r r r r r r r r}
$n$&$0$&1&2&3&4&5&6&7&8\\ \hline
$AD_n$& 1& 1& 3& 5& 7& 11& 19& 31& 49\\ \hline
$RS_n$& 1& 3& 5& 9& 17& 31& 57& 105& 193\\ \hline
$K_n$& 0& 1& 1& 2& 2& 3& 3& 4& 4\\ \hline

\end{tabular}$$


\bigskip \emph{Ricerca delle bijezioni disabilitata per questa stampa.}\par

\bigskip \subsection{Il problema} \



\bigskip Nel loro lavoro [Wilf], Wilf e Calkin basano la ricerca del numero di insiemi indipendenti di una supergriglia, $SG(m,n)$, sul concetto di \emph{matrice di trasferimento}, TM nel seguito. \par


\bigskip Il procedimento per costruire l'automa associato a questa supergriglia � il seguente\par
$$\mbox{$e$}\ \ \ 
\begin{tikzpicture}[,>=stealth',shorten >=1pt,auto,node distance=1.0cm,thick,main node/.style={circle,fill=blue!20,draw,font=\sffamily},main node2/.style={circle,fill=white!20,draw,font=\sffamily},main node3/.style={circle,fill=yellow!20,draw,font=\sffamily}] 
	\node[main node2] (0) {}; 
	\node[main node2] (1) [above of=0] {}; 
\path[every node/.style={font=\sffamily\small}] 
	(0) edge [] node[above] {} (1)
;
\end{tikzpicture}
\ \ \ 
\mbox{$u$}\ \ \ 
\begin{tikzpicture}[,>=stealth',shorten >=1pt,auto,node distance=1.0cm,thick,main node/.style={circle,fill=blue!20,draw,font=\sffamily},main node2/.style={circle,fill=white!20,draw,font=\sffamily},main node3/.style={circle,fill=yellow!20,draw,font=\sffamily}] 
	\node[main node2] (0) {}; 
	\node[main node] (1) [above of=0] {}; 
\path[every node/.style={font=\sffamily\small}] 
	(0) edge [] node[above] {} (1)
;
\end{tikzpicture}
\ \ \ 
\mbox{$d$}\ \ \ 
\begin{tikzpicture}[,>=stealth',shorten >=1pt,auto,node distance=1.0cm,thick,main node/.style={circle,fill=blue!20,draw,font=\sffamily},main node2/.style={circle,fill=white!20,draw,font=\sffamily},main node3/.style={circle,fill=yellow!20,draw,font=\sffamily}] 
	\node[main node] (0) {}; 
	\node[main node2] (1) [above of=0] {}; 
\path[every node/.style={font=\sffamily\small}] 
	(0) edge [] node[above] {} (1)
;
\end{tikzpicture}
$$

\bigskip Il sistema ottenuto dai possibili proseguimenti (di un passo) � il seguente:\par


$$\left\{
\begin{tabular}{ll} 
	$ee \longrightarrow\  e + u + d $\\
	$ed \longrightarrow\  e $\\
	$de \longrightarrow\  e + u $\\
	$eu \longrightarrow\  e $\\
	$ue \longrightarrow\  e + d $\\
\end{tabular}
\right.$$


\bigskip Riscriviamo lo schema con stringhe tutte di lunghezza $h$ (la potenza del cammino orizzontale):\par

$$\left\{
\begin{tabular}{ll} 
	$ee \longrightarrow\ ee + eu + ed $\\
	$ed \longrightarrow\ de $\\
	$de \longrightarrow\ ee + eu $\\
	$eu \longrightarrow\ ue $\\
	$ue \longrightarrow\ ee + ed $\\
\end{tabular}
\right.$$


\bigskip Un risultato di algebra lineare afferma che la somma degli elementi della matrice $(I-xTM)^{-1}$ � la funzione generatrice degli insiemi indipendenti. La matrice TM di questo esempio � \par

$$\begin{tabular}{|c|c|c|c|c|c|} \hline
$TM$   &  $ed$   &  $de$   &  $eu$   &  $ee$   &  $ue$\\ \hline
$ed$   &  $0$   &  $1$   &  $0$   &  $0$   &  $0$\\ \hline
$de$   &  $0$   &  $0$   &  $1$   &  $1$   &  $0$\\ \hline
$eu$   &  $0$   &  $0$   &  $0$   &  $0$   &  $1$\\ \hline
$ee$   &  $1$   &  $0$   &  $1$   &  $1$   &  $0$\\ \hline
$ue$   &  $1$   &  $0$   &  $0$   &  $1$   &  $0$\\ \hline
\end{tabular}$$ 

\bigskip La funzione generatrice �\par


$$F(x)=\frac{(-5 - 4x - 3x^2)}{(-1 + x + x^2 + x^3)}=5 +  9x +  17x^{2} +  31x^{3} +  57x^{4} +  105x^{5} + O(x^{6}) $$


\bigskip Dalla espansione in serie della fgo otteniamo i valori di $RS_n$ (\footnote{Ricordiamo che il metodo di Wilf non considera il grafo vuoto})\par

$$\begin{tabular}{c | r r r r r r r r r}
$n$&$0$&1&2&3&4&5&6&7&8\\ \hline
$RS_n$& 1& 3& 5& 9& 17& 31& 57& 105& 193
\end{tabular}$$


\bigskip Il coefficiente di $x^n$ nell'espansione in serie di questa funzione � il numero totale di insiemi indipendenti, dove $n$ � il numero di colonne del grafo considerato.\par
\subsection{Il nostro metodo} \



\bigskip Adesso costruiamo il sistema lineare in cui le variabili sono funzioni generatrici nell'indeterminata $x$. Avremo tante variabili ed equazioni quante sono le stringhe legali.\\ 
Alla generica linea dello schema 
$$ab \longrightarrow cd + \dots + ef $$ 
associamo la equazione 
$$AB(x) = xCD(x) +\dots + xEF(x) + 1$$ 

\par


\bigskip In questo caso abbiamo il seguente schema\par

$$\left\{
\begin{tabular}{ll}	$EE(x) =  x EE(x) +  x EU(x) +  x ED(x) +  1$\\
	$ED(x) =  x DE(x) +  1$\\
	$DE(x) =  x EE(x) +  x EU(x) +  1$\\
	$EU(x) =  x UE(x) +  1$\\
	$UE(x) =  x EE(x) +  x ED(x) +  1$\\
\end{tabular}
\right.$$



\bigskip L'automa che stiamo generando avr� uno stato per ogni stringa legale del linguaggio. Tutti gli stati sono finali. 
Ognuna delle nostre variabili � la funzione generatrice del linguaggio riconosciuto dall'automa a partire dallo stato corrispondente alla variabile.\\ 
In questo esempio lo stato iniziale � $EE$. Quindi risolvendo in $EE(x)$ si ottiene il linguaggio accettato dall'automa. 
\par


$$EE(x)=\frac{-((1 + x)^2}{(-1 + x + x^2 + x^3))}=1 +  3x +  5x^{2} +  9x^{3} +  17x^{4} +  31x^{5} + O(x^{6}) $$

$$\begin{tabular}{c | r r r r r r r r r r}
$n$&$0$&1&2&3&4&5&6&7&8&9\\ \hline
$RS_n$&1& 3& 5& 9& 17& 31& 57& 105& 193& 355
\end{tabular}$$


\bigskip abbiamo che il coefficiente di $x^t$ � il numero di insiemi indipendenti del grafo costituito dalle prime $t$ barrette verticali.\\\par


\bigskip Il software costruisce il sistema e genera l'automa \par

$$\left\{
\begin{tabular}{ll} 
	$EE \rightarrow e EE \mid  u EU \mid  d ED \mid   \lambda $\\
	$ED \rightarrow e DE \mid   \lambda $\\
	$DE \rightarrow e EE \mid  u EU \mid   \lambda $\\
	$EU \rightarrow e UE \mid   \lambda $\\
	$UE \rightarrow e EE \mid  d ED \mid   \lambda $\\
\end{tabular}
\right.$$

\bigskip 
\bigskip 
\begin{figure}[H]
	\centering
		\begin{tikzpicture}[>=latex',scale=0.8]
			\begin{dot2tex}[tikz] 
			digraph finite_state_machine { 
				graph[rankdir=LR, center=true, margin=0.2, nodesep=0.1, ranksep=0.3] 
				node[shape=doublecircle, fontname="Courier-Bold", fontsize=10, width=0.4, height=0.4, fixedsize=true] 
				edge[arrowsize=0.6, arrowhead=vee] 

				EE -> EE [ label = "e" ]; 
				EE -> EU [ label = "u" ]; 
				EE -> ED [ label = "d" ]; 
				ED -> DE [ label = "e" ]; 
				DE -> EE [ label = "e" ]; 
				DE -> EU [ label = "u" ]; 
				EU -> UE [ label = "e" ]; 
				UE -> EE [ label = "e" ]; 
				UE -> ED [ label = "d" ]; 
			} 
		\end{dot2tex} 
	\end{tikzpicture} 
\end{figure} 

\subsection{Ricorrenza Lineare} \



\bigskip In questo caso otteniamo la \emph{ricorrenza locale} dal denominatore della funzione generatrice della somma delle righe:\par


$$F(x)=\frac{(-5 - 4x - 3x^2)}{(-1 + x + x^2 + x^3)}\ .$$

$$\begin{tabular}{c | r r r r r r r r r}
$n$&$0$&1&2&3&4&5&6&7&8\\ \hline
$RS_n$&1&3&5&9&17&31&57&105&193
\end{tabular}$$
$$RS_n = RS_{n-1} + RS_{n-2} + RS_{n-3}$$ 


$$\mbox{schema}\ \ \ 
\begin{tabular}{| c | c |} \hline 
${1}$&${0}$\\ \hline 
${1}$&${0}$\\ \hline 
${0}$&${1}$\\ \hline 
${0}$&${*}$\\ \hline 
\end{tabular} $$ 

$$T(n,k)=T(n-1,k) + T(n-2,k-1) + T(n-3,k-1)$$\subsection{Antidiagonali} \

$$sigma = \begin{tabular}{| c | c |} \hline 
${1}$&${0}$\\ \hline 
${1}$&${0}$\\ \hline 
${0}$&${1}$\\ \hline 
${0}$&${*}$\\ \hline 
\end{tabular} $$ 

$$ H_{0} = 1, H_{1} = 1+2x, H_{2} = 1+4x $$ 

$$ V_{0} = (1-x)^-1 $$ 

 $$\begin{tabular}{| c |  c  c  c  c  c } \hline 
$1$&$0$&$0$&$0$&$0$&$0$\\ \hline 
$1$&$2$&$0$&$0$&$0$&$0$\\ \hline 
$1$&$4$&$0$&$0$&$0$&$0$\\ \hline 
$1$&$6$&$2$&$0$&$0$&$0$\\ 
$1$&$8$&$8$&$0$&$0$&$0$\\ 
$1$&$10$&$18$&$2$&$0$&$0$\\ 
\end{tabular} $$ 


\bigskip 
\bigskip 
\bigskip 
















































































 
\end{document}