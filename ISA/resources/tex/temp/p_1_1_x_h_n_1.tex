% Preamble

\documentclass{amsart}

%\usepackage{amssymb, amscd, amsfonts}
%\usepackage[mathscr]{eucal}
%\usepackage{graphicx}
\usepackage{lscape}
\usepackage{tikz}
\usepackage{scalefnt}
%\usepackage[applemac]{inputenc}
\usepackage[latin1]{inputenc}
%\usepackage[utf8x]{inputenc}
%\usepackage{verbatim}
%\usepackage[export]{adjustbox}
%\usepackage{lipsum}
\usepackage{float}% If comment this, figure moves to Page 2
\usepackage[pgf,outputdir={docgraphs/}]{dot2texi}
%\usepackage[active,tightpage]{preview}
\usetikzlibrary{shapes,automata,arrows}



%% Definitions of proclamation environments, commands, shorthands, et cetera

% Proclamation environments

\newtheorem{theorem}{Theorem}[section]
\newtheorem{lemma}[theorem]{Lemma}
\newtheorem{corollary}[theorem]{Corollary}
\newtheorem{proposition}[theorem]{Proposition}
\newtheorem{fact}[theorem]{Fact}

\theoremstyle{definition}
\newtheorem{observation}[theorem]{Observation}
\newtheorem{definition}[theorem]{Definition}
\theoremstyle{remark}
\newtheorem*{algoprob}{Problem}        % To typeset decision problems

\theoremstyle{remark}
\newtheorem*{algoinst}{Instance}       % To typeset decision problems

\theoremstyle{remark}
\newtheorem*{algoquest}{Question}       % To typeset decision problems

\theoremstyle{remark}
\newtheorem*{conjecture}{Conjecture}

\theoremstyle{definition}
\newtheorem{problem}{Problem}

\theoremstyle{remark}
\newtheorem*{question}{Question}

\theoremstyle{definition}
\newtheorem{remark}{Remark}

\theoremstyle{definition}
\newtheorem*{remarks}{Remarks}

\theoremstyle{definition}
\newtheorem{example}{Example}

\theoremstyle{remark}
\newtheorem{examples}{Examples}

\theoremstyle{definition}
\newtheorem{notation}[theorem]{Notation}

\theoremstyle{remark}
\newtheorem*{addnotation}{Additional Notation}

\theoremstyle{remark}
\newtheorem*{runningnotation}{Running Notation}

% Math Commands

\newcommand{\barecoeff}[2]{\gen\frac {} {} {0pt} {} {#1}{#2}}
\newcommand{\wstirl}[2]{\bigl\{\bigl\{\genfrac{}{}{0pt}{}{#1}{#2}\bigr\}\bigr\}}
\newcommand{\sstirl}[2]{ \genfrac {\{} {\}} {0pt} {} {#1}{#2} }
\newcommand{\wbinom}[2]{\bigl(\bigl(\genfrac{}{}{0pt}{}{#1}{#2}\bigr)\bigr)}


% Math Operators

\DeclareMathOperator{\fan}{\rm Fan}
\DeclareMathOperator{\chrom}{\chi} \DeclareMathOperator{\sign}{\rm
signum} \DeclareMathOperator{\partitio}{\rm p}
\DeclareMathOperator{\bbpartitio}{\rm p}
\DeclareMathOperator{\bbcappartitio}{\mathscr{P}}
\DeclareMathOperator{\Sub}{\rm Sub}
\DeclareMathOperator{\Parts}{\mathscr{P}}
\DeclareMathOperator{\MVParts}{\mathscr{MVP}}
\DeclareMathOperator{\Hom}{\textsc Hom}
\DeclareMathOperator{\Var}{\textxsc Var}
\DeclareMathOperator{\F}{\textxsc Form}
\DeclareMathOperator{\Eval}{\textsc Eval}
\DeclareMathOperator{\rank}{\rm rank}
\DeclareMathOperator{\pos}{\rm pos} \DeclareMathOperator{\lin}{\rm
lin} \DeclareMathOperator{\relint}{\rm relint}
\DeclareMathOperator{\con}{\rm conv}
\DeclareMathOperator{\aff}{\rm aff}
\DeclareMathOperator{\supp}{\rm supp}
\DeclareMathOperator{\eff}{\rm eff} \DeclareMathOperator{\st}{\rm
star} \DeclareMathOperator{\cstar}{\rm cstar}
\DeclareMathOperator{\lnk}{\rm link}
\DeclareMathOperator{\rel}{\rm relint}
\DeclareMathOperator{\Dom}{\rm Dom} \DeclareMathOperator{\Cod}{\rm
Cod} \DeclareMathOperator{\Ker}{\rm Ker}
\DeclareMathOperator{\Ran}{\rm Range}
\DeclareMathOperator{\GRan}{\rm GRange}
\DeclareMathOperator{\Dim}{\rm Dim}
\DeclareMathOperator{\vertices}{\rm vert}
\DeclareMathOperator{\Rad}{\mathscr{R}}
\DeclareMathOperator{\Spec}{\bf Spec}
\DeclareMathOperator{\Max}{\bf MaxSpec}
\DeclareMathOperator{\Zero}{\mathbf{Z}}
\DeclareMathOperator{\V}{\mathbb{V}}
\DeclareMathOperator{\I}{\mathbb{I}}
\DeclareMathOperator{\E}{\mathbb{E}}
\DeclareMathOperator{\A}{\mathbb{A}}
\DeclareMathOperator{\Aut}{\rm Aut}
\DeclareMathOperator{\FaceId}{\rm FaceId}
\DeclareMathOperator{\FaceR}{\rm FaceR}
\DeclareMathOperator{\GroupId}{\rm GroupId}
\DeclareMathOperator{\GroupR}{\rm GroupR}
\DeclareMathOperator{\den}{\rm den} \DeclareMathOperator{\lcm}{\rm
lcm}

\DeclareMathOperator{\tree}{\rm Tree}

% Commands

%% Temporary Annotations

\newcommand{\commento}[1]{\marginpar{\footnotesize \flushleft{ #1}}}

\newcommand{\remove}[1]{}

%% Sets of Numbers
\newcommand{\N}{{\mathbb{N}}}
\newcommand{\Prim}{{\mathbb{P}}}
\newcommand{\Z}{{\mathbb{Z}}}
\newcommand{\Q}{{\mathbb{Q}}}
\newcommand{\R}{{\mathbb{R}}}
\newcommand{\C}{{\mathbb{C}}}
\newcommand{\CC}{{\mathscr{C}}}
\newcommand{\lex}{{\overrightarrow{\oplus}}}
\newcommand{\MVN}{{\mathbb{M}\mathbb{V}\mathbb{N}}}
\newcommand{\MVZ}{{\mathbb{M}\mathbb{V}\mathbb{Z}}}
\newcommand{\MVQ}{{\mathbb{M}\mathbb{V}\mathbb{Q}}}
\newcommand{\MVR}{{\mathbb{M}\mathbb{V}\mathbb{R}}}
\newcommand{\MVC}{{\mathbb{M}\mathbb{V}\mathbb{C}}}

%% Frequent Bores
\newcommand{\asc}{\mathrm{asc}}
\newcommand{\des}{\mathrm{des}}

\newcommand{\QQ}{ \mathbb{Q} } % -- Rational numbers
\newcommand{\NN}{\mathbb{N}} % --- Natural numbers
\newcommand{\DD}{\mathcal{D}}
\newcommand{\LL}{\mathcal{L}}

\renewcommand{\SS}{\mathcal{S}}
\newcommand{\vbold}{\mathbf{v}}
\newcommand{\wbold}{\mathbf{w}}
\newcommand{\mss}{{\sf m}}
\newcommand{\M}{{\mathscr{M}}}
\newcommand{\G}{{\mathbf{\Gamma}}}
\newcommand{\Hats}{{\mathbf{H}}}
\newcommand{\lgp}{{$\ell$-group }}
\newcommand{\lgpO}{{$\ell$-group. }}
\newcommand{\lgps}{{$\ell$-groups }}
\newcommand{\lgpsO}{{$\ell$-groups. }}
\newcommand{\Fl}{\mathscr{A}}
\newcommand{\Fln}{\mathscr{A}_n}
\newcommand{\MV}{MV algebra }
\newcommand{\MVO}{MV algebra. }
\newcommand{\MVs}{MV algebras }
\newcommand{\MVsO}{MV algebras. }
\newcommand{\FMV}{\mathscr{M}}
\newcommand{\FMVn}{\mathscr{M}_n}
\newcommand{\Pro}{\textsc{BasProj}}
\newcommand{\In}{\textsc{Input: $\, \,$}}
\newcommand{\Out}{\textsc{Output: $\, \,$}}
\newcommand{\Y}{\textsc{Yes }}
\newcommand{\No}{\textsc{No }}
\newcommand{\pvar}{\vec{\pi}}
\newcommand{\li}{\langle \langle}
\newcommand{\ri}{\rangle \rangle}
\newcommand{\MM}{\mathcal{M}}
\renewcommand{\div}{{\, \, \rm div \,\,}}


%% Funny Names
\newcommand{\Lu}{{\L}u\-ka\-s\-i\-e\-w\-icz }
\newcommand{\Lus}{{\L}u\-ka\-s\-i\-e\-w\-icz's }
\newcommand{\Bez}{Bezout }

% Body
\begin{document}

\title{ISA Software v.1.3}

%%% Authors

%\author{O. M. D'Antona}

%%% Date

\date{\today}

%%% Addresses

%\address[O. M. D'Antona]{Dipartimento di Informatica e Comunicazione,
%via Comelico 39/41, I-20135 Milano, Italy}

%\email{dantona@dico.unimi.it}

\keywords{sample.tex}

%\subjclass[2000]{Primary: . Secondary: .}
%%% Abstract

%\begin{abstract}
%Evoluzione? 
%\end{abstract}

\maketitle 

%%% stampa test famiglie grafi

\section{ Caso di studio : Grafo $P_1^{( 1)}\times H_12^{( 1)}$  } 



\bigskip \begin{definition}
 Un grafo (non orientato e finito) � una coppia ordinata $(V,E)$ dove $V$ � un insieme finito ed $E$ � un multiinsieme di coppie non ordinate di elementi di $V$. L'insieme V contiene i vertici del grafo ed $E$ i suoi lati. Per un generico grafo $G$, l'insieme dei suoi vertici � indicato con $V(G)$ e quello dei suoi lati con $E(G)$.
 \end{definition} \par


\bigskip La struttura dati con la quale si � scelto di memorizzare il grafo � la matrice di adicenza.\par


\bigskip \begin{definition} 
La matrice di adiacenza di un grafo $G$ i cui vertici siano $v_1,v_2, \dots ,v_n$ � una matrice $A(G)=[a(i,j)]$  simmetrica di ordine $n\times n$ in cui si pone: 
\end{definition} 
$$a(i,j)=\left\{\begin{tabular}{ll} 
1& se $(v_i,v_j)\in E(G)$ \\ 
0& altrimenti 
\end{tabular}\right. 
 $$ 

\par


\bigskip Di seguito viene mostrata invece la lista di adiacenza che permette una pi� facile lettura delle adiacenze:\par


$$\left\{
\begin{tabular}{ll} 
	$(1;1) \longrightarrow\  (1;2), (1;12),  $\\
	$(1;2) \longrightarrow\  (1;1), (1;3),  $\\
	$(1;3) \longrightarrow\  (1;2), (1;4),  $\\
	$(1;4) \longrightarrow\  (1;3), (1;5),  $\\
	$(1;5) \longrightarrow\  (1;4), (1;6),  $\\
	$(1;6) \longrightarrow\  (1;5), (1;7),  $\\
	$(1;7) \longrightarrow\  (1;6), (1;8),  $\\
	$(1;8) \longrightarrow\  (1;7), (1;9),  $\\
	$(1;9) \longrightarrow\  (1;8), (1;10),  $\\
	$(1;10) \longrightarrow\  (1;9), (1;11),  $\\
	$(1;11) \longrightarrow\  (1;10), (1;12),  $\\
	$(1;12) \longrightarrow\  (1;1), (1;11),  $\\
\end{tabular}
\right.$$
$$\scalefont{0.3}\begin{tikzpicture}
		 [,>=stealth',shorten >=1pt,auto,node distance=1.4cm,thick,main node/.style={circle,draw,font=\sffamily\small}] 
	\node[main node] (1) {1.1}; 
	\node[main node] (2) [right of=1] {1.2}; 
	\node[main node] (3) [right of=2] {1.3}; 
	\node[main node] (4) [right of=3] {1.4}; 
	\node[main node] (5) [right of=4] {1.5}; 
	\node[main node] (6) [right of=5] {1.6}; 
	\node[main node] (7) [right of=6] {1.7}; 
	\node[main node] (8) [right of=7] {1.8}; 
	\node[main node] (9) [right of=8] {1.9}; 
	\node[main node] (10) [right of=9] {1.10}; 
	\node[main node] (11) [right of=10] {1.11}; 
	\node[main node] (12) [right of=11] {1.12}; 
 \path[every node/.style={font=\sffamily\small}] 	(2) edge  node[above] {} (1) 
	(3) edge  node[above] {} (2) 
	(4) edge  node[above] {} (3) 
	(5) edge  node[above] {} (4) 
	(6) edge  node[above] {} (5) 
	(7) edge  node[above] {} (6) 
	(8) edge  node[above] {} (7) 
	(9) edge  node[above] {} (8) 
	(10) edge  node[above] {} (9) 
	(11) edge  node[above] {} (10) 
	(12) edge  node[above] {} (11) 
(1) edge [bend left] node[above] {} (12); \end{tikzpicture}$$ 

\bigskip In forma circolare diventa:\par

\begin{figure}[H]

	\centering

		\begin{dot2tex}[circo,tikz] 
			strict graph { 
				 "1;1" -- "1;2"  ; 
				 "1;1" -- "1;12"  ; 
				 "1;2" -- "1;1"  ; 
				 "1;2" -- "1;3"  ; 
				 "1;3" -- "1;2"  ; 
				 "1;3" -- "1;4"  ; 
				 "1;4" -- "1;3"  ; 
				 "1;4" -- "1;5"  ; 
				 "1;5" -- "1;4"  ; 
				 "1;5" -- "1;6"  ; 
				 "1;6" -- "1;5"  ; 
				 "1;6" -- "1;7"  ; 
				 "1;7" -- "1;6"  ; 
				 "1;7" -- "1;8"  ; 
				 "1;8" -- "1;7"  ; 
				 "1;8" -- "1;9"  ; 
				 "1;9" -- "1;8"  ; 
				 "1;9" -- "1;10"  ; 
				 "1;10" -- "1;9"  ; 
				 "1;10" -- "1;11"  ; 
				 "1;11" -- "1;10"  ; 
				 "1;11" -- "1;12"  ; 
				 "1;12" -- "1;1"  ; 
				 "1;12" -- "1;11"  ; 
			} 
		\end{dot2tex} 

\end{figure} 

 

\bigskip Con le famiglie di grafi \emph{H} vogliamo indicare dei circuiti che hanno le potenze orizzontali \emph{limitate} al valore di $n$, quindi l'unico arco che fa da circuito � quello tra il primo nodo e l'ultimo.\par
\subsection{Calcolo insiemi indipendenti con metodo forza bruta} \



\bigskip \begin{definition}Un insieme indipendente di un grafo � un insieme di vertici non adiacenti del grafo.\end{definition}\par


\bigskip Definiamo $T(n,k)$ il numero di $k$-sottoinsiemi indipendenti di Grafo $P_1^{( 1)}\times H_12^{( 1)}$ . \\Ecco alcuni valori\par
$$\begin{tabular}{c | r r r r r r r}
$T(n,k)$&$k=0$&1&2&3&4&5&6\\ \hline 
$0$&1\\ 
$1$&1&1\\ 
$2$&1&2\\ 
$3$&1&3\\ 
$4$&1&4&2\\ 
$5$&1&5&5\\ 
$6$&1&6&9&2\\ 
$7$&1&7&14&7\\ 
$8$&1&8&20&16&2\\ 
$9$&1&9&27&30&9\\ 
$10$&1&10&35&50&25&2\\ 
$11$&1&11&44&77&55&11\\ 
$12$&1&12&54&112&105&36&2\\ 
\end{tabular}$$



\bigskip Seguono le successioni delle antidiagonali, della somma delle righe e dei valori massimali di $k$ per cui esistono insiemi indipendenti:\par

$$\begin{tabular}{c | r r r r r r r r r r r r r}
$n$&$0$&1&2&3&4&5&6&7&8&9&10&11&12\\ \hline
$AD_n$& 1& 1& 2& 3& 4& 5& 8& 12& 17& 25& 37& 54& 79\\ \hline
$RS_n$& 1& 2& 3& 4& 7& 11& 18& 29& 47& 76& 123& 199& 322\\ \hline
$K_n$& 0& 1& 1& 1& 2& 2& 3& 3& 4& 4& 5& 5& 6\\ \hline

\end{tabular}$$


\bigskip \emph{Ricerca delle bijezioni disabilitata per questa stampa.}\par


\bigskip \textbf{Wilf}: Non possiamo usare il metodo di Wilf per trovare la Fgo delle somme delle righe in quanto il grafo \`e un circuito.\par
\subsection{Automa} \

Questo \`e l'automa che riconosce (tutte e sole) le stringhe che corrispondono agli insiemi indipendenti di $C_n^{(1)}$: 

$$\begin{tikzpicture}[->,>=stealth',shorten >=1pt,auto,node distance=2.cm, 
thick,main node/.style={circle,fill=blue!20,draw,font=\sffamily},main node2/.style={circle,fill=green!20,draw,font=\sffamily},main node3/.style={circle,fill=yellow!20,draw,font=\sffamily}] 

		\node[initial,initial text=,state,accepting] (1) {S}; 
		\node[initial text=,state,accepting] (2) [right of=1] {$A$}; 
		\node[initial text=,state,accepting] (3) [ below of=1] {$B$};  
		\node[initial text=,state,accepting] (4) [right of=2] {$E_1$}; 
		\node[initial text=,state,accepting] (5) [right of=4] {$E_2$}; 
		\node[initial text=,state,accepting] (6) [right of=3] {$F_1$}; 
		\node[initial text=,state] (7) [right of=6] {$F_{2}$}; 
		\node[initial text=,state,accepting] (8) [right of=7] {$G$}; 

		\path[every node/.style={font=\sffamily\small}] 

		(1) edge[] node[above] {$e$} (2) 
		(1) edge[] node[] {$a$} (3) 
		(2) edge[loop above] node[left=1pt] {e} (2) 
		(3) edge[] node[below] {$e$} (6) 
		(6) edge[loop below] node[left=1pt] {e} (6) 
		(5) edge[bend right] node[above] {$a$} (4) 
		(5) edge[loop above] node[left=1pt] {e} (5) 
		(2) edge[] node[] {$a$} (4) 
		(4) edge[] node[] {$e$} (5) 
		(6) edge[] node[above] {$a$} (7) 
		(7) edge[] node[] {$e$} (8) 
		(8) edge[bend left] node[below] {$a$} (7) 
		(8) edge[loop above] node[left=1pt] {e} (8); 

\end{tikzpicture}$$ 


\bigskip Il sistema lineare diventa:\par
\bigskip

\noindent\(\pmb{\text{eqs}=\{}\\ 
\pmb{s\text{==}t*a+t*b+1,}\\ 
\pmb{a==t*a+t*\text{e1}+1,}\\ 
\pmb{\text{e1}==t*\text{e2}+1,}\\ 
\pmb{\text{e2}==t*\text{e1}+t*\text{e2}+1,}\\ 
\pmb{b==t*\text{f1}+1,}\\ 
\pmb{\text{f1}==t*\text{f1}+t*\text{f2}+1,}\\ 
\pmb{\text{f2}==t*g,}\\ 
\pmb{g==t*\text{f2}+t*g+1}\\ 
\pmb{\}}\) 


\bigskip $$F(t)=\frac{-1-t+t^3}{-1+t+t^2}$$ 

\bigskip $$1+2 t+3 t^2+4 t^3+7 t^4+11 t^5+18 t^6+29 t^7+47 t^8+76 t^9+O[t]^{10}$$ 



\bigskip Calcolo automatico sistema lineare e automa per circuiti:\par
$$\mbox{$e$}\ \ \ 
\begin{tikzpicture}[,>=stealth',shorten >=1pt,auto,node distance=1.0cm,thick,main node/.style={circle,fill=blue!20,draw,font=\sffamily},main node2/.style={circle,fill=white!20,draw,font=\sffamily},main node3/.style={circle,fill=yellow!20,draw,font=\sffamily}] 
	\node[main node2] (0) {}; 
\path[every node/.style={font=\sffamily\small}] 
;
\end{tikzpicture}
\ \ \ 
\mbox{$u$}\ \ \ 
\begin{tikzpicture}[,>=stealth',shorten >=1pt,auto,node distance=1.0cm,thick,main node/.style={circle,fill=blue!20,draw,font=\sffamily},main node2/.style={circle,fill=white!20,draw,font=\sffamily},main node3/.style={circle,fill=yellow!20,draw,font=\sffamily}] 
	\node[main node] (0) {}; 
\path[every node/.style={font=\sffamily\small}] 
;
\end{tikzpicture}
$$

$$\left\{
\begin{tabular}{ll} 
	$e \longrightarrow\  e + u $\\
	$u_i \longrightarrow\  e_f $\\
	$u \longrightarrow\  e $\\
	$s \longrightarrow\  e + u_i $\\
	$u_1 \longrightarrow\  e_f $\\
	$e_f \longrightarrow\  e_f + u_1 $\\
\end{tabular}
\right.$$

$$\left\{
\begin{tabular}{ll}	$E(x) =  x E(x)  +   x U(x)  + 1 $\\
	$U(x) =  x E(x)  + 1 $\\
	$U_i(x) =  x E_f(x)  + 1 $\\
	$S(x) =  x E(x)  +   x U_i(x)  + 1 $\\
	$U_1(x) =  x E_f(x)   $\\
	$E_f(x) =  x E_f(x)  +   x U_1(x)  + 1 $\\
\end{tabular}
\right.$$


$$\left\{
\begin{tabular}{ll} 
	$E \rightarrow e E  \mid  u U  \mid \lambda $\\
	$U \rightarrow e E  \mid \lambda $\\
	$U_i \rightarrow e E_f  \mid \lambda $\\
	$S \rightarrow e E  \mid  u U_i  \mid \lambda $\\
	$U_1 \rightarrow e E_f   $\\
	$E_f \rightarrow e E_f  \mid  u U_1  \mid \lambda $\\
\end{tabular}
\right.$$


$$E(x)=\frac{(-1 - x + x^3)}{(-1 + x + x^2)}=1 +  2x +  3x^{2} +  4x^{3} +  7x^{4} +  11x^{5} + O(x^{6}) $$

$$\begin{tabular}{c | r r r r r r r r r r}
$n$&$0$&1&2&3&4&5&6&7&8&9\\ \hline
$RS_n$&1& 2& 3& 4& 7& 11& 18& 29& 47& 76
\end{tabular}$$

\bigskip 
\bigskip 
\begin{figure}[H]
	\centering
		\begin{tikzpicture}[>=latex',scale=0.8]
			\begin{dot2tex}[tikz] 
			digraph finite_state_machine { 
				graph[rankdir=LR, center=true, margin=0.2, nodesep=0.1, ranksep=0.3] 
				node[shape=doublecircle, fontname="Courier-Bold", fontsize=10, width=0.4, height=0.4, fixedsize=true] 
				edge[arrowsize=0.6, arrowhead=vee] 

				S -> "E" [ label = "e" ]; 
				S -> "U_i" [ label = "u" ]; 
				U -> E [ label = "e" ]; 
				U_i -> E_f [ label = "e" ]; 
				E -> E [ label = "e" ]; 
				E -> U [ label = "u" ]; 
				U_1 -> E_f [ label = "e" ]; 
				E_f -> E_f [ label = "e" ]; 
				E_f -> U_1 [ label = "u" ]; 


 				 U_1 [shape=circle];			} 
		\end{dot2tex} 
	\end{tikzpicture} 
\end{figure} 


\bigskip 
\bigskip 
\bigskip 
















































































 
\end{document}